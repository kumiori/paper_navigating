%!TEX root = main.tex

\section{Discussion} 
\label{sec:discussion}
% \added[id=ALB]{
    Capturing branch switching phenomena across stability transitions is not an automatic feature of approximate numerical methods. If they rely on approximate information about the Hessian of the energy functional, these methods do not guarantee to systematically detect transitions between critical equilbrium states, when stability is lost. 
Indeed, this requires a careful determination of the zero-eigenmodes that render singular the exact nonlinear Hessian, which is typically not available in general purpose first order numerical algorithms.
In practice, without such information, critical loads for equilibrium transitions become algorithm-dependant and are not consistent with closed-form solutions of the exact evolution problem.
% }
% 
% -------
% \begin{figure}[htbp]
%     \centering
%     \includegraphics*[width=.7\textwidth]{../images/irreversibility_inf_eig.png}
%     \caption{\footnotesize{ \textcolor{blue}{LABELS ON AXIS} Stability of the homogeneous solution in the first model on (a) the linear space and (b) on convex cone $K^+_0$. When the irreversibility constraint is present, the homogeneous solution is stable for any value of the loading parameter $\bar\epsilon$.}}
%     \label{fig:shouldnt}
% \end{figure}


Our evidence is that certain numerical methods can introduce non-physical artifacts which should be distinguished from genuine physical phenomena. Our ongoing work aims to refine numerical techniques to provide more reliable algorithms for analysing irreversible processes in variational evolutionary problems with multiple local minima and a high number of degrees of freedom.

From our numerical experience, first order solutions to strongly nonlinear, nonconvex, and singular problems, like those of interest in the applications exhibit strong sensitivity to numerical errors, possibly leading to spurious bifurcations and artificial state transitions. On the other hand, solutions which inegrate second order information are robust and their observability can be fully characterised.

More than numerical perturbations (which can always arise,) the use of numerical methods relying only on (conjugate) gradients (in lieu of exact Hessians) is prone to introducing non-physical crack nucleation.

This is an important observation, which highlights the need for a thorough investigation of the stability of solutions. 
If only physical factors are considered, an energetic selection mechanism is already encapsulated in the stability statement in the evolution law. As a consequence, equilibrium solutions under increasing load should be maintained as observable only if stable, assuming that no nucleation should occur otherwise.

We present two main options for discussing these considerations in view of the importance of using accurate and robust algorithms in real scenarios.

\begin{itemize}
    \item 
    Ignore the Numerical Artifact: focussing solely on first order considerations and acknowledging that the observed computed nucleations may be purely numerical and should not be considered in physical terms.
    \item 
    Highlight the Numerical Artifact: Alternatively, emphasizing that homogeneous solutions should be observable, despite 
    the sensitivity to numerical parameters (artifacts, in the quasi-Newton approach) and the abundance of admissible solutions (in the nonconvex scenario).
    \item 
    Otherwise.
\end{itemize}

Suggesting that state transitions in complex scenarios should be carefully interpreted, the connection between observability and stability is functional to understanding real patterns that emerge, e.g., in higher dimensions or in other physical systems.

In either case, our computations show that, unless second order analysis is performed, observed nucleations are not \emph{necessarily} indicative of physical cracks but rather of an interplay between purely   \textcolor{black}{physical} phenomena, inherent to the nature of natural processes, and numerical biases inherent to the computational methods employed. This distinction is crucial for understanding the limitations and proper application of numerical techniques as predictive tools in contexts where cracks are a real concern for structures.

Future work will include a more detailed exploration of evolutionary algorithms and their implementation for stability analysis of fracture in thin films. Some notable instances are craquelures artistic paintings~\cite{fuster-lopez:2020-picassos, Bosco:2020aa,Bosco:2021},  brittle stability of the cryosphere~\cite{weiss:2017-linking, tollefson:2017-giant, Sun:2023aa, Millan:2023aa} and crack-pattern selection in metallic thin films~\cite{Faurie2019-to}. \textcolor{black}{As understood in this work, the final crack patterns depend heavily on the form of the selected unstable mode. In turn, the irreversibility constraint, primarily implemented to prevent crack closure, directly impacts the stability of solutions and mode selection. Therefore, our findings suggest that the success of the phase-field model in predicting crack patterns in thin films relies on the use of robust numerical algorithms.}

% PERFORMING AN THOROUGH characterisation of bifurcation points, conditions for stability exchange, local features of the energy landscape, qualitative features of constrained systems with respect to canonical representations of bifurcations, thorough TECHNIQUES OF energy blow up and reduction at bifurcation points.



