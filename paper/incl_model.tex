%!TEX root = main.tex
\section{Material, Structure, and Evolution}
\label{sec:rigid}

Two one-dimensional fracture models for a brittle thin film bonded to substrates with different mechanical properties provide a framework to investigate the evolution and stability of crack patterns under external loading, in a  simple scenario where multiplicity of solutions, equilibrium bifurcations, and stability transitions interplay.


\paragraph{Material Model}
We consider a one-dimensional isotropic and homogeneous brittle material modelled by a state function $W(e, \alpha, \alpha')$ which, at any point $x$,
% The displacement field of the thin film is denoted by $u(x)$ and the corresponding strain field is $\epsilon(x) = u'(x)$.
depends on the local membrane strain $e(x)$ (associated to in-plane displacements $u(x)$, namely $e(x) = u'(x)$), the local damage $\alpha(x)$, and the local gradient of the damage $\alpha'(x)$. 
Here, the damage variable $\alpha$ is a scalar field driving material softening, bounded between zero and one, where $0$ indicates the undamaged material and $1$, the cracked material. Thus, at points where $\alpha=0$ the material is elastic with a stiffness $\mathsf{E}$ (its Young modulus), at points where $\alpha=1$ the material has a crack and zero residual stiffness, whereas for intermediate damage values the material's stiffness is $0<\mathsf{E}\soften(\alpha)< \mathsf{E}$.  
The state function $W$ is defined as 
\begin{equation}
    \label{def:energy_material}
    W(e, \alpha, \alpha'):= \frac{1}{2} \mathsf{E} \soften(\alpha)e^2 + W_1\homogdiss(\alpha) + \frac{\damagell^2}{2}\alpha'^2,
\end{equation}
where $\homogdiss(\alpha)$ can be interpreted as the energy dissipated during an homogeneous damaging process, compounds a term that is proportional to the square of its gradient and controls the energy cost of spatial damage variations. In the first summand, $\soften(\alpha)$ is the function that describes the material softening. %
For physical consistency, $\soften(\alpha)$ is a non-negative function that is one when $\alpha=0$ and monotonically decreases as $\alpha\to 1$, reaching zero for $\alpha=1$  
On the other hand, $\homogdiss(\alpha)$ is a non-negative, zero only if $\alpha=0$, and monotonically increases with $\alpha$, reaching $\homogdiss(1)=1$. 
The damage-dependent stress is $\sigma(\alpha):=\mathsf{E}a(\alpha) e$. The parameter $\damagell$ is a characteristic length that controls the competition between localization and homogeneous damage, effectively controlling the width of damage localizations, the peak stress of the material in one-dimensional traction experiments, and - more in general - structural size effects.
Specifically, both functions $\soften(\alpha)$  
% of the form $\soften(\alpha) = (1-\alpha)^2$
 and $\homogdiss(\alpha)$ are chosen to be quadratic, namely 
%  $\homogdiss(\alpha) = \mathsf{w_1}\alpha^2$.
\begin{equation}
    \label{def:constitutive_functions}
    \soften(\alpha) = (1-\alpha)^2, \quad \homogdiss(\alpha) = \mathsf{w_1}\alpha^2,
\end{equation}
This modelling choice is common (yet not unique) in phase-field fracture models (cf.~\cite{Bourdin2000-pc,Miehe2010-sj,Miehe2010-ja}). In the current context, it allows damage to evolve for an arbitrarily small value of the load. 
%  consequences on the damage material behaviour and the properties of the evolutions.




\paragraph{Structural Model}
The structure under consideration is a multilayer composite constituted by a brittle thin film  made of the material identified by the state function $W$, attached to an underlying substrate which is either rigid or elastically compliant. The thin film is a one-dimensional membrane with thickness $h$ and length $L$ with $L\gg h$, subject to a combination of imposed inelastic tensile strains $\bar\epsilon_t$, imposed displacements by the substrate, and loadings at the boundary. The structure's reference configuration is the interval $(0, L)$. 
% 
The substrate is modelled as a one-dimensional elastic foundation WHOSE displacement field IS $v(x)$ AND strain field is $e(v) = v'(x)$. 
% 
The displacement field which is elastically compatible to an homogeneous strain in the substrate, is the linear function $v(x, t) = \bar\epsilon(t)/2 (2x-1)$, where $\bar\epsilon(t)$ is the applied tensile strain and $t$ plays the role of a loading time parameter. 
% 
The film is subjected to $v(x, t)$ as a tensile imposed load and to given (compatible) displacements at its free ends $x = 0, L$, so that for all $t$, $u(0)=v(0, t)$ and $u(L)=v(L, t)$
% 
Our first model describes a brittle thin film deposited on a stiff, non-deformable substrate. This model assumes the substrate is rigid, meaning, $v$ is a given. The elastic interaction is modelled by a distributed linear elastic foundation of stiffness $K$, thus
% , so that the energy of the structure is given by
the total energy of the structure is a functional $\Estiff$ constructed by considering the energy of the thin film and the energy associated to the mechanical coupling between the film and the substrate. In nondimensional form, it is given by

\begin{equation}
    \label{def:energy_stiff}
    \Psi(\alpha, u) = \int_{0}^{1} \left[ \frac{1}{2} \ \soften(\alpha)(u')^2 
    + \frac{1}{2 \elastell^2} (u-\subsu)^2
    + \homogdiss(\alpha) + \frac{\damagell^2}{2}(\alpha')^2 
     \right] dx,
\end{equation}
where $\Lambda^2 = \frac{E_{\text{eff}}}{K}$, $E_{\text{eff}}$ being the effective stiffness of the two dimensional membrane.
Our second model involves the same brittle thin film but a \emph{compliant} elastic substrate that undergoes deformation alongside the film. Unlike for the rigid substrate, the substrate's deformation is an additional unknown which affects the overall energy landscape of the system, incorporating an extra term accounting for the strain energy of the substrate. 
The state of this structure is identified by the triplet $y:=(u, \alpha, v)$, and the nondimensional energy of the compliant system  $\Ecompl(\alpha, u, v)$ reads

\begin{equation}
    \label{def:energy_compliant}
    \Ecompl(\alpha, u,  v) = \int_{0}^1 \left[ \frac{1}{2} \ \soften(\alpha)(u')^2 + \homogdiss(\alpha) + \frac{\damagell^2}{2}(\alpha')^2 
    + \frac{1}{2 \elastell^2} (u- v)^2 
    + \frac{\stiffratio}{2}  (v')^2 \right] dx,
\end{equation}


% Dimensional parameters

Nondimensional constants appearing in the expression of the energies can be related the three-dimensional parameters of a phyisical mechanical system by an appropriate de-scaling, remarking that the reduced elastic model~\eqref{def:energy_stiff} can be obtained as an asymptotic limit starting from a thin multilayer system in 3d elasticity, as shown in~\cite{Leon_Baldelli2015-rp} .
% , \mathsf{E'_\text{2d}}
Thus, denoting by $\mathsf{E_\text{2d}}$ the two-dimensional stiffness of the thin film membrane (per unit depth), and by a prime sign the stiffness and thickness of the substrate, the nondimensional quantities appearing in the expression of the energy depend on the geometric the elastic properties of the system as follows
\begin{equation}
    \label{def:dimensional_parameters}
    \damagell := \frac{\bar\damagell}{L}, \quad
    \Lambda := \frac{\Youngfilm}{\Youngsubs}\frac{h w_1}{L^2}, \quad
    \stiffratio := \frac{\Youngfilm}{\Youngsubs}\frac{w_1}{h}, \quad
\end{equation}
while spatial variables and physical displacements are respectively normalized with respect to the film's length $L$ and the displacement scale $u_0 := \frac{\mathsf{w_1}L}{\mathsf{E_{\text{2d}} h}}$.




% Functional setting

Denoting for brevity $y:=(u, \alpha)$ (and $y:=(u, \alpha, v)$) the mechanical state of the stiff (respectively, compliant) model system, as well as by $H^1(0, 1)$ the (Sobolev) space of scalar functions defined on the unit interval which are square integrable and have square integrable first derivatives. The energy $\Estiff(y)$ is well-defined for pairs belonging in the Cartesian vector product space $V:=H^1(0,1)\times  H^1(0,1)$. Similarly, the energy $\Ecompl(y)$ is well-defined for triplets $(u, \alpha, v)$ in $\widetilde V:=V\times  H^1(0,1)$. In both cases, we extend the value of the total energy to $+\infty$ whenever $\alpha < 0$ or $\alpha > 1$ without renaming the functional.
% The energy functionals are coercive, lower semi-continuous, and nonconvex.

% \subsection{Thin Film on Stiff Substrate}
% \label{sec:thin_film_stiff}

% %  The energy functional for this system is constructed by considering the elastic energy of the thin film and the energy associated with the phase-field variable that describes the damage state of the film.


% The total energy functional $\Estiff(\alpha, u)$ in the nondimensional form is given by:


% where $u(x)$ is the displacement field of the brittle membrane and 
% $\epsilon(x)=u'(x)$ the corresponding continuum strain variable, \added[id=ALB]{$\damagell, \elastell$\comment[id=ALB]{Notation: $\ell, \ell_1$} are two nondimensional lengthscales. The former identifies the peak material stress XXXXX in a one dimensional test as well as the typical scale of the damage localization bands. The latter, quantifies the film's stiffness relative to the substrate}. 
% We \replaced[id=ALB]{specify the material model by choosing constitutive functions}{make the standard assumptions for constitutive functions}  $\soften(\alpha) =(1-\alpha)^2$ and $\homogdiss(\alpha) =\alpha^2$ and thus our model corresponds to the  phase field fracture model referred to as the Ambrosio and Tortorelli (AT2) model in. 
% We also choose the boundary conditions in the form: $u(0)=-\bar\epsilon_t/2$, $u(1)=\bar\epsilon_t/2$ and we impose the displacement field  of the substrate as \added[id=ALB]{a loading mode parametrized by $t$,} $\subsu(x, t)=\bar \epsilon_t/2 (2x-1)$, where $\bar \epsilon_t = \bar \epsilon t$ describes a monotonic loading history.



% \added[id=ALB]{nondimensional quantities: $\elastell = , \damagell, G_c,\sigma_c, ...$}
% \subsection{Thin Film on Compliant Substrate}
% \label{sec:thin_film_compliant}

% The total energy functional for this system is given by:

% where $r_{\rm s}$\comment[id=ALB]{check notation} is the ratio between elastic constants of substrate and thin film and $ v$ is the displacement field on the deformable non-breakable substrate with $\epsilon_{\rm s}= v'(x)$ represents the deformation field of the elastic foundation. We  choose boundary conditions in the form: $u(0)=-\bar\epsilon/2$, $u(1)=\bar\epsilon/2$,  $ v(0)=-\bar\epsilon/2$, $ v(1)=\bar\epsilon/2$ and, 

% in the case of the compliant substrate, the displacement of the substrate is an additional unknown field variable, elastically coupled to the film, and the energy of the structure is given by





\paragraph{Evolutionary model}
\label{sec:stability}

Assuming small enough loading rates, 
the evolution problem of the structure can be cast in an energetic variational formulation as an incremental rate-indipendent quasi-static process driven by an energy-minimality principle.
This allows identyfing sequences of equilibrium configurations as critical states, to ascertain their stability, and determining the system's transition trajectories between different equilibrium states.
Optimality conditions defining this problem are derived from the
intuitive idea that a state is observable (if and) only if it is stable, and in turn, a state is stable only if it is a local minimum of the energy among admissible state perturbations, at fixed load. 

Differently from dissipative evolutions driven by an energy gradient flow whereby a system \emph{reaches} equilibrium conditions through a gradient descent process parametrized by an internal timescale, the quasi-static evolution we consider is a sequence of \emph{attained} equilibrium states as a necessary condition for local energy minimality. 
The rate-indipendency further implies that the system does not exhibit internal timescales.
As such, energy minimisation is performed at any given value of the load, and at each increment of the external load its configuration evolves subject to imposed boundary conditions and possible internal constraints.
% With the assumption that the loading process is slow enough for inertia and dynamic effects to be negligible, this evolutionary model 
% 
% so that the damage process is quasi-static, and the evolution of the damage variable is governed by the principle incremental energy minimality, stating that the state of the system evolves in such a way that the total energy of the system is minimised, subject to the imposed loads, among admissible configurations (or, equivalently, with respect to admissible perturbations).

% In this work, 
Specifically, we consider an evolution during the loading interval $t\in [0, T]$ as a time-parametrized mapping $t\mapsto y_t$  such that, for all $t\in [0, T]$ the actual state of the system $y_t$ is a local energy minimum among all admissible state perturbations, or with respect to all admissible competitor states. %
In practice, given an initial condition $y_0$ at $t=0$ we seek a state $y_t:=(u, \alpha)_t$ (and $\widetilde{y_t}:=(u, \alpha, v)_t$ respectively, for the compliant model) such that, for a given value of the control parameter $t$, it satisfies time-dependent kinematic boundary conditions on the displacement variable and is locally energy minimal. 
For definiteness, denoting by $X_t = \{\varphi\in H^1(0, 1):\varphi=\subsu(x, t),\text{ for } x=0\text{ and } x=1\}\times H^1(0,1)$ the affine vector space of (kinematically) admissible states for the stiff model, (respectively $\widetilde{X_t} = X_t\times H^1(0, 1)$, for the compliant model) 
we seek 
% $y_t\in X_t$ such that
\begin{equation}
    \label{eq:variational_global_ineq}
    % \Estiff(y_t) \leq \Estiff(y_t + z),\quad \forall \text{ admissible } z,
    y_t\in X_t: \qquad \Estiff(y_t) \leq \Estiff(y),\quad \forall \text{ admissible competitors } y,
\end{equation}
and similarly for the compliant model substituting $\Estiff$ with $\Ecompl$ and $X_t$ with $\widetilde{X_t}$.
% for the stiff model, and  $y_t\in \widetilde{X_t}$ such that
% \begin{equation}
%     \label{eq:variational_global_ineq}
%     \Ecompl(y_t) \leq \Ecompl(y_t + z),\quad \forall \text{ admissible } z,
% \end{equation}
\paragraph{Admissibility of competitors and perturbations}
The admissibility of state competitors explicitly depends on the loading parameter through the kinematic boundary conditions (on displacement) and on internal constraints, namely whether damage (and hence the softening material behaviour)  evolves in a reversible or irreversible manner.
In the first case, as damage can evolve freely within the interval $[0, 1]$, all admissible states in $X_t$  (respectively, in $\widetilde{X_t}$) are also admissible competitors. 
In the second case, the damage field is subject to the pointwise irreversibility constraint $\dot \alpha(x)\geq 0, \forall x\in [0, 1]$, requiring that the damage can only increase or stay constant. 
As a consequence, irreversibility restricts the admissible set of competitors to the set $K^+_{\alpha_t}:=H^1(0, 1) \times \{\beta\in H^1(0, 1): \beta\geq \alpha_t\}$ (respectively, $\widetilde{K^+_{\alpha_t}}:=K^+_{\alpha_t}\times H^1(0, 1)$). 
Remark that in the definition of the competitor space $\alpha_t$ (the damage field at time $t$) is unknown at time $t$. 
In the irreversible case the set of admissible competitor states depends explicitly on the entire history of the evolution through the current damage field.
To draw the attention to the consequences of irreversibility on the system's transitions between different equilibrium states we develop the global variational inequality~\eqref{eq:variational_global_ineq} (and the analogous for the compliant model) by expanding the energy around the state $y_t$. 
Admissible perturbations in the fully reversible case they belong to the linear space $X_0$ associated to $X_t$ (respectively, $\widetilde{X_t}$), whereas in the irreversible case they are bound to the closed, pointed, convex cone $K^+_0$ (respectively, $\widetilde{K^+_0}$).

An energy expansion in the vicinity the state $y_t$ reads
$$
\Estiff(y)-\Estiff(y_t)= \delta\Estiff(y_t)(y-y_t)+\frac{1}{2}(y-y_t)^T \delta^2\Estiff(y_t)(y-y_t)+o(\|y-y_t\|^2)
    \label{eqn:energy-expansion}
$$
% and is valid for any $y$ whose displacement component verifies kinematic boundary conditions and its damage component $\alpha$ is either unconstrained, or such that $\alpha\geq\alpha_t$ if irreversibility is taken into account.
(and analogously for the compliant model), leading to write first and second order necessary (and sufficient) conditions for optimality.


% Local Stability: For each load step, the system is in a stable equilibrium, meaning no small admissible perturbation can decrease the system's energy.

% Irreversible Processes: The principle is often used to model processes with irreversible changes, such as damage and fracture, where the damage variable (e.g., ($\alpha$)) can only increase or stay constant.

In the subsequent section, we exploit the energy-stability inequality~\eqref{eq:variational_global_ineq} for both mechanical models introduced above, exploring the equilibrium configurations and identifying the conditions under which the system transitions between different equilibrium branches. 
This analysis will involve both analytical and numerical methods, with a focus on understanding equilibrium bifurcations and stability properties of these phase-field fracture models.



%  we consider the phase-field theory of fracture to describe the crack formation of {a thin elastic membrane} attached to a substrate. The evolution equation for the phase-field is determined by the minimization of an energy functional based on the variational theory of brittle fracture~\cite{bourdin:2008-the-variational}.
% In phase-field theories of fracture, elasticity of the brittle solid is usually assumed to be linear with a stiffness degrading with damage. The latter is described by a scalar order parameter which features in the total energy functional with one term that accounts for the energy dissipation associated to homogeneous damage, and with 

% Suppose that the scalar damage variable $\alpha(x)$ is \replaced[id=ALB]{normalised such that}{with} $\alpha=0$ and $\alpha=1$ respectively correspond to the unbroken an fully broken states, \added[id=ALB]{and that the system's evolution is parametrized by a positive scalar control parameter, henceforth noted $t$, which can be thought of as a kinematic time and with respect to which loads are indexed}.


% 

\section{Linear and nonlinear stability}
\label{sec:stability}

Solutions to the incremental evolutionary problem are sought by solving first and second order necessary conditions for optimality encoded in the global variational inequality~\eqref{eq:variational_global_ineq} which reduces to a \emph{linear} stability problem in a linear vector space for the case of fully reversible damage, and constitutes an instance of a \emph{nonlinear} stability problem in a convex cone in the presence of irreversibility.

\paragraph{Linear stability in the reversible case - stiff substrate}

% In the expression above $z$ is a state perturbation, its admissibility requires zero boundary conditions for the displacement component as well as additional constraints on its damage component when irreversibility is taken into account. 
% Define $X_t = H^1(0, 1)$
% Provided a given initial condition $y_0$ at $t=0$, the quasi-static, incremental, evolutionary problem we address is to find a parametrized map $t \mapsto y_t \in X_t$, such that~\eqref{eq:variational_global_ineq} is satisfied.


% PERTURBATIONS, DIRECTIONAL DERIVATIVES, CONSEQUENCES OF MINIMALITY

Equilibrium equations of the system are obtained as first order necessary conditions for energy minimality, satisfying imposed displacements at the ends of the film.

The first order variation of the energy functional $\Estiff$ in the direction  \comment[id=OUS]{change the notation of directions to make them compatible with the rest} $z:=(w, \beta)$ is given by the following linear form  
\begin{equation}
    \delta \Estiff(u,\alpha)(w,\beta)=\int_0^1
\left[\soften(\alpha)u'w'+\frac{1}{\elastell^2} (u-u_0) w+ \left( \frac{1}{2}u'^2 \soften'(\alpha)+\homogdiss'(\alpha) \right)\beta+\damagell^2\alpha'\beta' \right]dx,\label{firstvar1}
\end{equation}
where 
% $v$ and $w$ are scalar test functions. 
$z\in X_0$ is a test function (an admissible perturbation) in the linear space associated to $X_t$. First order minimality conditions are \emph{local} conditions that require the stationarity or the energy functional, that is, its first order variation should vanish for all admissible test functions, namely
\begin{equation}
    \label{eq:stationarity}
    \delta \Estiff(u,\alpha)(w,\beta)=0, \quad \forall z\in X_0.
\end{equation}
% Note here that, because the damage field $\alpha$ is free from irreversibility constraits, both positive and negative  test functions  are admissible, thus equilibrium configurations necessarily satisfy the equality to zero of the first order variation of the energy functional. 
By using standard arguments of the calculus of variations, localizing the integral and choosing $\beta = 0$ first, and then $v =0$ leads to establishing the strong form of local equilibrium conditions which couple the mechanical equilibrium and the damage criterion, respectively given by
\begin{eqnarray}
    % \label{modeld_el_1}
\begin{cases}
  2(1-\alpha)\alpha' u' +(1-\alpha)^2 u'' -  \frac{1}{\elastell^2}(u-\subsu) &= 0, \quad {x\in (0, 1)}\\
  -\damagell^2\alpha'' - (1-\alpha)( u')^2 + 2\alpha   &= 0, \quad {x\in (0, 1)}
\end{cases}
\label{auto1}
\end{eqnarray}
% proceeding with the evaluation of the equilibrium configuration via the Euler-Lagrange equations
This differential system is equipped with essential kinematic boundary conditions $u(0) = \subsu(0, t), u(1) = \subsu(1, t)$. Conversely, conditions on the internal field $\alpha'(0)=\alpha'(1)=0$ naturally follow by minimality.
% When the order parameter is reversible, the space of admissible perturbations is the homogeneous space associated to $X_t$, that is $V_0:= H^1_0((0,1))\times  H^1((0,1))$.
Notice that the choice of boundary conditions for displacements compatible with the substrate's deformation implies that the pair $\yhom:=(u_h(x), \alpha_h)$ given by $u_h(x)\equiv \subsu(x, t)$ and $\alpha_h$ a load-dependent constant to identify, is always (the unique homogenous) solution to the first order equilibrium equations. This makes it immediate to identify the fundamental homogeneous solution branch $t \mapsto \yhom$ and to decouple the elasticity problem from the  evolution of damage. 
Therefore, the solution to~\eqref{auto1} such that $u''(x)=\alpha'(x)= 0, \, \forall x\in (0, 1)$ is the homogeneous branch
%  $y_h = (u_h, \alpha_h)$ whose  read 
\begin{equation}
u_h(x) =  \frac{\bar \epsilon_t}{2}(2x-1),\qquad\alpha_h = \frac{\Bar{\epsilon_t}^2}{2 + \Bar\epsilon_t^2}\label{eq:homo1}.
\end{equation}
Notice that,  
first order conditions identify the the critical load threshold that activates the damaging process. namely, the quantity $\bar\epsilon^*$ is the value of the load for which the pristine elastic state becomes unstable. It is determined injecting the elastic solution $u_h$ in the energy and computing $\delta \Estiff(u_h, 0)(0, \beta)=0$.
According to our energy, the critical load is given by
\begin{equation}
    \bar{\epsilon}_*^2 = 2\frac{\homogdiss'(0)}{\soften'(0)} 
\end{equation}
% 
which, by our choice of material model (whereby $\homogdiss(\alpha)$ is quadratic, cf~\eqref{def:constitutive_functions}) we get that  \comment[id=OUS]{aligned bar over epsilon} $\bar{\epsilon}_*=0$ so that damage necessarily starts as soon as $t>0$.
The effective {total} energy along the {homogeneous} branch reads  
\begin{equation}
    \label{eq:energy_homogeneous}
    \Estiffhom(\bar\epsilon) :=\Estiff(\yhom) = \frac{\Bar{\epsilon}^2}{2 + \Bar\epsilon^2}.
\end{equation}
An equilibrium configuration $y_t:=(u,\alpha)_t$ is a state such that the first variation $\delta \Estiff(y_t)(z)$ vanishes for all admissible test fields in the vector space $X_0$. 
To assess the incremental stability of the homogeneous solution in the reversible (linear) case, we examine the positivity of the second variation, requiring
\begin{equation}
\delta^2 \Estiff(y_t)(y_t-z, y_t-z)>0, \qquad  \forall z\in X_0,
\label{eqn:linear_second_order_stability}
\end{equation} 
{The second directional derivative of the energy is given by the following bilinear form}
\begin{equation}
\delta^2 \Estiff(u,\alpha)(v,\beta)=\int_0^1 \left[(1-\alpha)^2v'^2 
+\frac{1}{\elastell^2} v^2 \right]dx
%+ \elastell^{-2}v^2
+\int_0^1
\left[ - 4(1-\alpha)u' v'\beta+(2+ u'^2)w^2+\damagell^2\beta'^2 \right] dx, 
\label{hessian22}
\end{equation}
% \comment[id=ALB]{notation $w\mapsto \beta$} 
which is well defined for perturbations $(v, \beta)\in X_0$.
{We} extract information on the onset of instability  seeking a solution in Fourier series of the fields $v$ and $\beta$ in \eqref{hessian22}, such that $v(x)=\sum_{n=1}^{\infty} a_{n} \sin \left(n \pi x+\phi_{n}\right), \quad \beta(x)=\sum_{n=1}^{\infty} b_{n} \cos \left(n \pi x+\psi_{n}\right)$. Then, we observe that, thanks to boundary conditions, $\psi_{n}=\phi_{n}=0$ for all natural $n$. The stability condition~\eqref{eqn:linear_second_order_stability}  takes the form:
\begin{align}\left[ a_n \quad b_n \right] \mathcal{H} \left[ \begin{array}{c} a_n \\ b_n \end{array} \right]=\left[ a_n \quad b_n \right]\left(
\begin{array}{cc}
\soften(\alpha)(n\pi)^2+\elastell^{-2}  & \frac{\partial \soften}{\partial \alpha}\bar\epsilon(n\pi)  \\
\frac{\partial \soften}{\partial \alpha}\bar\epsilon(n\pi)  &   \frac{\partial^2 \soften}{\partial \alpha^2}\bar\epsilon^2+ \frac{\partial h^2}{\partial \alpha^2}+\damagell^2(n\pi)^2  \\
\end{array}
\right)\left[ \begin{array}{c} a_n \\ b_n \end{array} \right]>0.\label{hessian1}\end{align}
% 
By substituting the homogeneous solution $\alpha_h$ into \eqref{hessian1}, we compute $\det \mathcal{H}$ as a function of $\bar \epsilon$ and $n$, which is depicted in Figure~\ref{fig:hessian1}. The calculations are performed for the parameter values $\damagell=0.16$ and $\elastell=0.34$.
    % Interpretation of the image: 
The figure represents, for a given load $\bar \epsilon$, the wave number $n(\bar \epsilon)\in \mathbb N$ of possible energy-decreasing damage bifurcations. For an increasing loading history $\bar \epsilon_t\nearrow$, the wave number is non-monotonic.
In Fig. \ref{fig:hessian1}(a), the locus $\det \mathcal H=0$ forms closed loops, indicative of an elastic background's influence\comment[id=ALB]{Expand}. Notably, a re-entry behavior of the affine configuration is discernible, marked by the emergence of two critical strains denoted as $\bar\epsilon^*$ and $\bar\epsilon^{**}$ (with $\bar\epsilon^* < \bar\epsilon^{**}$), representing the lower and upper stability limits for the homogeneous state. These critical points are highlighted by red and green dots in Fig. \ref{fig:hessian1}(a). The critical wave number $n_c$ for the lower limit $n_c(\bar{\epsilon}^*)$    differs from the critical wave number for  upper limit $n_c(\bar{\epsilon}^{**})$. Finally, we remark that closed-form analytical solutions can be provided for the critical wave number and critical strains and the parametric dependence of the corresponding bifurcation thresholds can be obtained, as detailed in \cite{Salman2021-mn}.

\begin{figure}
     \centering
    %  \begin{overpic}[width=\linewidth]{your_image_file.pdf}
        % \includegraphics[scale=0.25]{./final_images/fig1.pdf}
        \begin{overpic}[width=\linewidth]{./final_images/fig1.pdf}
            % (0,0) at the bottom-left corner and (100,100) at the top-right corner.
        \put(33, 35){\textcolor{white}{$\bar \epsilon^{**}$}} % Adjust the coordinates to position the text as needed
        \put(88, 35){\textcolor{white}{$\bar \epsilon^{**}$}} % Adjust the coordinates to position the text as needed
        \put(59, 10){\textcolor{white}{$\bar \epsilon^*$}} % Adjust the coordinates to position the text as needed
        \put(7, 10){\textcolor{white}{$\bar \epsilon^*$}} % Adjust the coordinates to position the text as needed
    \end{overpic}
\caption{
% \todo[inline]{Add markers for $\bar \ep^{*}, \bar \ep^{**}$}
Computation of the determinant $\det \mathcal{H}$ for the homogeneous solution $\alpha_h$, utilizing parameter values $\damagell=0.16$ and $\elastell = 0.34$. The computation is performed for two scenarios: (a) rigid foundation and (b) compliant foundation with the additional parameter $\stiffratio =0.5$. The closed loops observed indicate a re-entry behavior for large deformations. Black dots mark bifurcation from the homogeneous solution while red and green dots indicate critical strains $\bar{\epsilon}^*$ and $\bar{\epsilon}^{**}$, representing the lower and upper stability limits, respectively. The critical wave number $n_c$ for the lower limit $n_c(\bar{\epsilon}^*)$ differs from that for the upper limit $n_c(\bar{\epsilon}^{**})$.}
     \label{fig:hessian1}
 \end{figure}

 \paragraph{Linear stability in the reversible case - compliant substrate}



Denoting by $z:=(v, \beta, \subsutest)\in \widetilde{X}_0$ a test function for the state triplet  $y_t:=(u, \alpha, \subsu)_t$ at time $t$, the first order variation of the energy functional $\Ecompl$ is given by the following linear form 
% The first variation of the energy functional is given by
\begin{equation}
\delta \Ecompl(u, \alpha, \subsu)(v,\beta,\subsutest)=\int_0^1 [(1-\alpha)^2u'v'+\frac{1}{\elastell^2} (u- v) (v- \subsutest)+\stiffratio \subsu'\subsutest'+\frac{1}{2}u'^2 (\soften'(\alpha)+\homogdiss'(\alpha))\beta+\damagell^2\alpha'\beta' ]dx,\label{firstvar}
\end{equation}
% where $v$, $\beta$ and $z$ are  test functions. 
The Euler-Lagrange equations read
\begin{eqnarray}
\label{modeld_el_1}
\begin{cases}
  2(1-\alpha)\alpha' u' +(1-\alpha)^2 u'' -  \frac{1}{\elastell^2}(u-\subsu) &= 0, \\
  -\damagell^2\alpha'' - (1-\alpha)( u')^2 + 2\alpha   &= 0,\\
    \stiffratio  \subsu''  +  \frac{1}{\elastell^2}(u-\subsu) &= 0. \\
\end{cases}
\label{auto2}
\end{eqnarray}
It is easy to show that the homogeneous solution on the trivial branch remains the same as in the case of the rigid substrate, that is $\alpha_h(\bar{\epsilon}) = \frac{\bar{\epsilon}^2}{2 + \bar{\epsilon}^2}\label{eq:homo11}$, whereas the effective elastic energy along the homogeneous branch now reads $\Ecomplhom(\bar{\epsilon}) = \frac{\bar{\epsilon}^2}{2 + \bar{\epsilon}^2} + \frac{\rho}{2}\bar{\epsilon}^2$.

We once more seek the linear incremental stability of an equilibrium configuration $y_t := (u, \alpha,  v)_t$ satisfying that the first order condition $\delta \Ecompl(y_t)(y-y_t)=0$ for all admissible state perturbations $z:=y-y_t=\in\widetilde X_0$ by examining the positivity of the second variation, namely
\begin{equation*}
    (y-y_t)^T\delta^2 \Ecompl(y_t)(y-y_t)>0, \qquad \forall y-y_t\in \widetilde X_0,
\end{equation*}The second variation is given by the following bilinear form 
\begin{equation}
\delta^2 \Ecompl(y_t)(y, y)=\int_0^1 \left[(1-\alpha_t)^2v'^2 
%+ \elastell^{-2}v^2
- 4(1-\alpha_t)u_t' v'\beta+(2+ {u_t'}^2)\beta^2+\damagell^2\beta'^2 +\frac{1}{\elastell^2}( v^2 + \subsutest^2)+\stiffratio \subsutest'^2\right]dx.\label{hessian222}\end{equation}
We again proceed to extract information on the onset of instability expanding in Fourier series the fields $v$, $w$ and $z$ appearing in \eqref{hessian222} such that 
\[
v(x) = \sum_{n=1}^{\infty} a_{n} \sin \left(n \pi x + \phi_{n}\right),
\quad \beta(x) = \sum_{n=1}^{\infty} b_{n} \cos \left(n \pi x + \psi_{n}\right),
\quad \subsutest(x) = \sum_{n=1}^{\infty} c_{n} \sin \left(n \pi x + \theta_{n}\right).
\]

Similar to the previous section we claim that, whenever the first order term vanishes, the system is stable only if $\delta^2 \Ecompl
% (u,\alpha, v)(v,\beta,\subsutest)(v,\beta,\subsutest)
>0$ for all sufficiently smooth admissible test fields $(v, \beta, \subsutest)$ in the vector space $\widetilde{ X_0}$. The stability condition  takes the form
\begin{align}
    % \left[ a_n \quad b_n \quad c_n  \right] \mathcal{H} \left[ \begin{array}{c} a_n \\ b_n  \\ c_n\end{array} \right]=
    % 
    \left[ a_n \quad b_n \quad c_n \right]\left(
\begin{array}{ccc}
\soften(\alpha)(n\pi)^2+\elastell^{-2}& \frac{\partial \soften}{\partial \alpha}\bar\epsilon(n\pi) &  -\elastell^{-2} \\
\frac{\partial \soften}{\partial \alpha}\bar\epsilon(n\pi) & \frac{\partial \soften^2}{\partial \alpha^2}\bar\epsilon^2+ \frac{\partial \homogdiss^2}{\partial \alpha^2}+\damagell^2(n\pi)^2 & 0 \\
 -\elastell^{-2} & 0 & \elastell^{-2} +\stiffratio (n\pi)^2\\
\end{array}
\right)\left[ \begin{array}{c} a_n \\ b_n \\c_n \end{array} \right]>0.\label{hessian3}\end{align}
By substituting the homogeneous solution $\alpha_h$ into \eqref{hessian3}, we compute $\det \mathcal{H}$, as depicted  in Fig. \ref{fig:hessian1}(b). The calculations are performed for the parameter values $\damagell=0.16$, $\elastell=0.34$ and $\rho=0.5$.  In Fig. \ref{fig:hessian1}(b), we again observe the formation of closed loops and a re-entry behavior of the affine configuration is discernible  (i.e., the homogeneous configuration re-stabilizes at large deformation). This is marked by the emergence of two critical strains denoted as $\bar\epsilon^*$ and $\bar\epsilon^{**}$, representing the lower and upper stability limits for the homogeneous state. These critical points are highlighted by red and green dots in Fig. \ref{fig:hessian1}(b). The critical wave number $n_c$ for the lower limit $n_c(\bar{\epsilon}^*)$    differs from the critical wave number for the upper limit $n_c(\bar{\epsilon}^{**})$. The overall behavior of the system remains the same while the first critical wave number $n_c=2$ is now smaller suggesting a different number of cracks will appear during the loading history.




% 

\paragraph{Nonlinear stability with irreversibility constraints}

Irreversibility is introduced in the variational formulation of the evolution problem as a pointwise inequality constraint. Intuitively, irreversibility plays two distinct roles along an evolution: it acts (i) as a \emph{local} constraint which prevents the damage field at a given location to decrease between two subsequent loads values (both during monotonic and non-monotonic load programs), and (ii) as a global restriction of the space of admissible variations in such a way that \emph{negative} perturbations of the current damage state are no longer allowed. This changes the topological structure of the set of admissible perturbations, from a linear vector space to a convex cone. 
To enforce irreversibility we consider only non-decreasing damage evolutions that are sufficiently smooth with respect to the loading parameter, and seek maps $t\mapsto y_t = (u_t, \alpha_t)$ such that $\dot \alpha_t \geq 0$ and that satisfy the minimality condition~\eqref{eq:variational_global_ineq}.
Because the current state can only be compared to those of with equal or higher damage, the space of admissible perturbations is a  convex cone strictly contained in the vector space of admissible unconstrained perturbations, namely $K^+_0\subset X_0$. Indeed, for $(v, \beta)\in K^+_0$, then $-(v, \beta) \in X_0$ but $-(v, \beta)\notin K^+_0$. This restriction has a profound impact on the variational characterisation of local minima and bears consequences on both for the first order (equilibrium) conditions and the second order (stability) problem which become \emph{unilateral} conditions.
% namely such that $\dot \alpha_t(x) \geq 0 \text{ a.e. } x\in (0, 1)$,
Equilibrium states $y_t=(u_t, \alpha_t)$ of the irreversible system, are hence governed by the following (first order) necessary optimality conditions taking the form of a variational inequality
\begin{equation}
    \label{eq:eq_variational_inequality_full}
    \delta \Estiff(y_t)(y-y_t) \geq 0, \quad \forall y-y_t \in K^+_0.
    % \delta \Psi(y)(v-u, w-\alpha) \geq 0, \quad \forall
    %  w -\alpha \in \{ w\in H^1(\Omega), w\geq 0\}
\end{equation}
% 
which has to hold for all admissible competitor states $y$ such that $v-u_t \in H^1_0(0, 1)$ and $\beta-\alpha_t\in H^1(0,1)$ with $\beta \geq \alpha_t$.
% where $K^+_{t}:=\subsu(t) + H^1_0(0, 1)\times\{w \in H^1(0, 1): w(x) \geq \alpha_t(x) \text{ a.e. }x\in (0, 1)\}$ depends on the loading parameter $t$ both explicitly, in the definition of kinematic boundary conditions, and implicitly, through the pointwise damage irreversibility.
% Notice tha
By testing the elasticity problem for fixed damage, and the damage problem for a given displacement field, we obtain 
\begin{equation}
    \label{eq:variational_equilibrium} 
    \delta \Estiff(y_t)(v-u_t, 0) = 0, \qquad \delta \Estiff(y_t)(0, \beta-\alpha_t) \geq 0, \qquad \forall y-y_t \in K^+_{0}.
\end{equation}

% \added[id=ALB]{
The last two relations are, respectively, the weak form of the mechanical equilibrium conditions and the  evolution law  for the damage field.
The former allows to compute the elastic equilibrium displacement $u_t$ which can thus be eliminated from energy, while the latter governs the evolution of the damage field.
Upon elimination of the kinematic field, the variational inequality~\eqref{eq:variational_equilibrium}.2 takes a particularly expressive form when written as a complementarity problem. This highlights the mechanical nature of the damage criterion as a threshold law.
Namely, the strict convexity of the elastic model for given damage implies that~\eqref{eq:variational_equilibrium}.1 has, for given $\alpha$, a unique time-parametrized solution $u_t(\alpha)$. 
Substituting in ~\eqref{eq:variational_equilibrium}.2 and accounting for the irreversibility constraint we are led to seek a map $t\mapsto\alpha_t$ such that 
% }
    \begin{equation}
    \label{eq:complementarity}
    \dot \alpha_t \geq 0 \qquad 
     -\phi_t(\alpha_t) \leq 0 \qquad
     \phi_t(\alpha_t)\dot \alpha_t = 0.
\end{equation}
Here, $\phi_t$ is the scalar function associated to the variation of elastic energy density at the equilibrium, defined by $\delta \Estiff(y_t)(0, \beta) = \langle -\phi_t(\alpha_t), \beta\rangle$. Here, the subscript $t$ indicates that this quantity is computed for the equilibrium $u_t$.
%  Here, $\phi_t$ is the energy variation with respect to perturbations of the order parameter (from a given damage level $\alpha_t$) and is interpreted as an energy release rate. 
Consequently, $\phi_t(0)$ is 
% the total energy released by the system from its sound state
the variation of the  energy density at equilibrium for the undamaged structure, and all equilibrium solutions $u_t$ such that $-\phi_t(0) > 0$ belong to the interior of the damage yield surface for the sound structure. The equality $-\phi_t(0) = 0$, conversely, indicates that the damage criterion has been attained by the sound structure, or equivalently, that the state $(u_t, 0)$ has reached, from the interior, the boundary of the (damage-dependent) elastic domain. 
% 
% 
% 
Explicitly, the function $\phi_t$ depends on the brittle material model and is defined, for both the stiff and compliant substrate models, as
\begin{equation}
    \label{eq:energy_release_rate}
    \phi(\alpha) := \frac{1}{2}\soften'(\alpha)\epsilon^2 + \homogdiss'(\alpha),
\end{equation}
where $\epsilon$ is the elastic strain and $\soften'(\alpha)$ and $\homogdiss'(\alpha)$ are the derivatives of the softening and dissipation energy densities with respect to the damage variable.
The inequality ~\eqref{eq:variational_equilibrium}.2 identifies the domains of admissible strains (and, by duality, of stresses) for homogeneous solutions as
$\mathcal{R}(\alpha):=\{e: \Youngfilm \epsilon^2 \leq -\frac{2 \mathsf{w}'(\alpha)}{\mathsf{a}'(\alpha)}\},$ and $
\mathcal{R}^*(\alpha):=\{\sigma: \frac{\sigma^2}{\Youngfilm} \leq \frac{2 \mathsf{w}'(\alpha)}{\mathsf{s}'(\alpha)}\}$, respectively.
% 
% The inequality $-\phi_t(\alpha_t) \leq 0$ 
As a first order optimality condition,~\eqref{eq:variational_equilibrium}.2 states that the local elastic energy release is either smaller than or equal to the (marginal) cost of damage, whereas the complementarity condition $\phi_t(\alpha_t)\dot \alpha_t = 0$ ensures that the damage field evolves only if the energy release rate is critical.
The three conditions above, established as necessary first order condition for constrained optimality, encode the pointwise non-negativity of the damage rate, the boundedness of the elastic domain, its dependence upon damage, and the complementarity between the attainment of the damage criterion and the conditions for the evolution of the internal order parameter. 

In the current one-dimensional setup with homogeneous initial conditions $y_0=(0, 0)$ and compatible kinematic boundary conditions, the existence of a homogeneous solution implies that the damage criterion is attained everywhere throughout the bar at the same load. This greatly simplifies the analysis of the energetic properties of the system. 
Using the elastic solution $u_t = 2\bar \epsilon_t(x-1/2)$ 
% and denoting by $\delta \alpha:=w-\alpha_t$ an admissible damage perturbation,
 the inequality in~\eqref{eq:variational_equilibrium}.2 yields the following algebraic inequality
\begin{equation}
    \label{eq:variational_equilibrium_homogeneous}
    0 \geq -\phi_t(\alpha_t)= -\frac{1}{2}\bar \epsilon_t^2 \soften'(\alpha_t)-\homogdiss'(\alpha_t).
\end{equation}
% \comment[id=ALB]{} 
% The equality in the last condition corresponds to the attainment a critical energy release rate whereby the elastic energy released (the negative quadratic term) equals (and drives) damage evolution.
% We have $\delta \Psi(y_t)(0, \delta\alpha) =0, \, \forall \delta \alpha \geq 0$ only if
% $$
% \frac{1}{2}t^2 g'(\alpha_t)+\homogdiss'(\alpha_t) = 0, \qquad \text{a.e. } x\in (0, 1).
% $$
The equality in \eqref{eq:variational_equilibrium_homogeneous} is an algebraic equation for $\alpha_t$ which identifies the evolution of the homogeneous damage response, as a function of the given load level $t$.
The investigation of the stability properties requires  
considering second order energy variations with respect to all admissible perturbations that render null the first order term in the energy expansion~\eqref{eqn:energy-expansion}.
In the general case, this requires distinguishing between the regions where the damage criterion is attained, and thus damage can evolve, from the (complementary) domain damage cannot evolve (there, the second relation in~\eqref{eq:complementarity} is satisfied with a strict inequality).
In our setup, the existence of nontrivial homogeneous solutions simplifies the analysis because the damage criterion is attained everywhere, damage can increase throughout the whole domain, and the function space of admissible perturbations is defined on the fixed domain $(0, 1)$.

% as the non-negativity of the hessian form on the constrained space of admissible perturbations is not guaranteed by the positive definiteness of the quadratic form on the homogeneous linear space generated by $K^+_0$.  


Assume now that a state $y_t$ is known as a function of $t$ such that it solves~\eqref{eq:eq_variational_inequality_full} and is sufficienty smooth so that the (right) derivative with respect to $t$ is well-defined. 
As $t$ varies, $y_t$ describes a (smooth) curve in the phase space identified by its right tangent vector $\dot y_t =: \lim_{\tau\to 0^+}\frac{y_{t+\tau} - y_t}{\tau}$, the rate of evolution.
A fundamental question is to discern whether $y_t$ is an isolated equilibrium state tracing a unique evolution path, or conversely if it lays at the intersection of multiple equilibrium curves.
% two curves can intersect transversally or tangentially, there is an an- gular or tangent bifurcation following from the fact that the tangent directions to these curves are different or not at a bifurcation point.multiplicity
% The problem of finding all equilibrium curves going through a given equilibrium point is ... due to the nonlinearities of the problem. 
% Restricting our attention to evolutions which are smooth with respect to the loading parameter $t$, 
To this end, differentiating~\eqref{eq:eq_variational_inequality_full} with respect to $t$ we  obtain a boundary value problem relating the rate of evolution $\dot y_t$ to the current state $y_t$, supposing the latter known, namely
% 
\begin{equation}
    \label{eq:variational_bifurcation}
    % \text{find }\dot \zeta: 
    \qquad \delta^2 \Psi(y_t)(\dot y_t,   \zeta -\dot y_t) + \delta \dot \Psi(y_t)(\zeta-\dot y_t) \geq 0, \quad \forall \zeta\in X_{0},
\end{equation}
% \comment[id=ALB]{explain the notation $K^+_{\dot y_t}$}
where $\delta \dot \Psi$ is the time-derivative of the linear form corresponding to the first order energy variation.
By construction, the homogeneous rate $\dot y_h$ is a solution of \eqref{eq:variational_bifurcation}, the question is whether another solution exists. The uniqueness is thus ensured by the positive definiteness of the quadratic form 
% within the homogeneous linear space generated by $K^+_0$, namely
in $X_0$. 
Thus, the non-bifurcation condition for the homogeneous evolution reads
% . The bifurcation problem is then governed by the following variational inequality
% - The bifurcation problem which is established by derivation of the first order equilibrium problem [.]
\begin{equation}
    \label{eq:bifurcation_uniqueness}
    \delta^2 \Psi(\yhom)(\zeta, \zeta) > 0, \quad \forall \zeta \in X_0,
\end{equation}
which formally coincides with the classical \emph{linear stability} problem~\eqref{eqn:linear_second_order_stability} in the reversible case, yet has a different mechanical interpretation in terms of evolution rates and uniqueness of equilibrium branches.
Remark that, in the general case in which the damage criterion is not attained everywhere, the space of admissible perturbations for the (second order) bifurcation problem is $X_0':=H^1_0(0, 1) \times \{ \beta \in H^1(0, 1) : \delta\Psi(y_t)(0, \beta) = 0 \}$ for the stiff substrate model, and $\widetilde{X_0'}:=X_0'\times H^1_0(0,1)$ for the compliant substrate model. 
The first load at which the bifurcation inequality~\eqref{eq:bifurcation_uniqueness} fails, namely $t_b:=\inf_t \{\delta^2 \Psi(y_h)(\zeta, \zeta) =0, \forall \zeta \in X_0 \}$ corresponds to the first  bifurcation load, namely, the load for which there exist (multiple) equilibrium curves intersecting the homogeneous branch. As a consequence, for $t\geq t_b$  the possibility exists of bifurcating away from homogeneous branch. The study of the bifurcation problem is functional to infer a partial response on the stability of the state. Indeed, if the current equilibrium branch is unique then, necessarily, the current state is stable. The converse is not true, however, as the existence multiple possible of bifurcation paths is not a sufficient condition exclude the stability of the current state. This holds true for the irreversible case, due to the conceptual difference between the bifurcation and the stability problems.

The stability of the homogeneous solution in the irreversible case, according to our energetic viewpoint, is governed by the positivity of $\delta^2 \Psi(\yhom)$ on the constrained space of admissible state perturbations $K^+_0$.
% , yet the converse is not true.
Denoting $t_s$ the load at which the homogeneous solution looses stability by analogy to the bifurcation load, namely $t_s: = \inf_t \{\delta^2 \Psi(\yhom)(\zeta, \zeta) =0, \forall \zeta \in K^+_0 \}$, the set inclusion $K^+_0 \subset X_0$ implies that necessarily $t_b \leq t_s$. (equality occurs when the first bifurcation mode is nonnegative)

This indicates a qualitative conceptual distinction between the bifurcation and the stability thresholds, in the irreversible case. As a consequence, a system can persist along a critical non-unique equilibrium branch, yet be stable. 
A sufficient condition for the stability of the homogeneous state $\yhom$ is given by the strict positivity  of the Hessian form, on the constrained space of admissible perturbations, namely (for the stiff substrate model)

\newcommand{\conespace}{K^+_{0}}
% As a straightforward consequence,  
% Indeed, the critical load for loss of stability is conceptually different from the critical load at which the uniqueness of the (rate) response is lost, they are determined by two different problems which can be related by a (singular) asymptotic process.
\begin{equation}
     \label{eq:variational_stability}
     \delta^2 \Psi(\yhom)(y - \yhom,  y - \yhom)  > 0, \quad \forall y-\yhom \in \conespace,
 \end{equation}
and similarly for the compliant substrate model, by replacing $\Estiff$ with $\Ecompl$ and $\conespace$ with $\widetilde{\conespace}$.
The variational inequality above is a constrained eigenvalue problem and a tool to characterize the stability of the state $\yhom$. Its solution yields either a positive eigenpair $(\kappa, z^*)\in \mathbb{R}^+\times K^+_0$ as a sufficient condition for the stability of current state, or a pair $(\kappa, z^*)\in \mathbb{R}^-\times K^+_0$ where $\kappa$ is the local (negative) energy curvature and the eigenmode $z^*$ indicates the direction of maximum energy decrease, interpreted as the \emph{instability mode} which points the system towards an optimal direction of energy descent. 
From the numerical standpoint, the bifurcation eigen-problem in the vector space~\eqref{eq:variational_bifurcation} may be regarded as an approximated version of the stability problem, in the time-discrete setting. As suggested in~\cite{Baldelli2021-gc} through the notion of `incremental-stability', the irreversibility constraint in the stability problem can be relaxed to a pointwise inequality with respect to the state at the previous time-step, denoted $y_-$.
This allows to replace the \emph{current} state $y_t$ in the definition of the space of perturbations, with $y_-$. As a consequence, assuming that $\alpha_t$ is the equilibrium damage field solving first order optimality conditions at $t$ and that $\alpha_-$ is the solution at the previous load step, admissible perturbations for the second order problem~\eqref{eq:variational_stability} are all the $y-y_-\in \conespace$. In this way, the set of perturbations is enlarged. It includes all sufficiently smooth functions $\beta$ which cancel the first order term, without restriction on the sign provided that $\alpha_t(x) + \beta(x) - \alpha_-(x)\geq 0$ for all $x\in (0, 1)$.
% 
% $\alpha$ is admissible at $t$ only if $\alpha \in H^1(0, 1)$ and $\alpha \geq \alpha_{-}$ where $\alpha_{-}$ is the damage state at the previous load step. 
In this way, the space of perturbation allows for (small) perturbations $\beta$ which can be negative where $\alpha_t(x)>\alpha_-(x)$. Such a space is a vector space and the eigen-problem can be solved by standard methods of linear algebra by restricting the Hessian to the set of active constraints, cf~\cite{Nocedal1999-zr}.
Conversely, the stability problem~\eqref{eq:variational_stability}, a constrained eigenvalue problem in a convex cone, is of a different nature altogether due to the different topology of the underlying energy space. 
The associated discrete problem can be numerically solved by exploiting the orthogonality between the set $\conespace$ and its dual $K^*:=\{y \in H^1(0,1): \langle x,y \rangle \leq 0, \forall x \in \conespace\}$, cf.~\cite{Moreau1962-fz,Pinto_da_Costa2010-qv}. \comment[id=OUS]{corrected}

% \begin{equation}
    % H_{\conespace} = \{z\in X_0: \delta^2 \Psi(y_h)(z, z) \geq 0\}.
% \end{equation}




% Consequently, without modyfing the vector structure space of perturbations.
