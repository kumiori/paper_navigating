%!TEX root = main.tex
\section{Model: thin film on stiff substrate}
\label{sec:rigid}
In this work, we consider the phase-field theory of fracture to describe the crack formation of \replaced[id=ALB]{a thin elastic membrane}{an elastic bar} attached to a substrate. The evolution equation for the phase-field is determined by the minimization of an energy functional based on Griffith's theory of brittle fracture.
In phase-field theories of fracture, elasticity of the breaking solid is usually assumed to be linear with stiffness degrading with damage. The latter is described by a scalar order parameter with the square of the gradient of this parameter controlling the energy cost of the damage non-affinity\comment[id=ALB]{Suggest: spatial damage variations}.
Suppose that the scalar damage variable $\alpha(x)$ is \replaced[id=ALB]{normalised such that}{with} $\alpha=0$ and $\alpha=1$ respectively correspond to the unbroken an fully broken states, \added[id=ALB]{and that the system's evolution is parametrised by a positive scalar control parameter, henceforth noted $t$, which can be thought of as a kinematic time}. We can write the energy of the system in the \added[id=ALB]{nondimensional} form\comment[id=ALB]{Suggest: a hint on nondimensional rescalings}
\begin{equation}
\label{modeld}
\Psi(\alpha, u) = \int_{0}^{1} \left[ \frac{1}{2} \ g(\alpha)(u')^2 + h(\alpha) + \frac{\lambda_1^2}{2}(\alpha')^2 
+ \frac{1}{2 \lambda_2^2} (u-u^s)^2 \right] dx,
\end{equation}
where $\epsilon(x)=u'(x)$ the continuum strain variable and $u(x)$ is the corresponding displacement field, \added[id=ALB]{$\lambda_1, \lambda_2$ are two nondimensional lengthscales, the former identifying the peak material stress as well as the typical scale of damage localisation bands and th latter quantifying the film's stiffness relative to the substrate.
}. 
\added[id=ALB]{According to the expression above, the energy $\Psi$ is well defined for elements of the vector space $V:=H^1((0,1))\times  H^1((0,1))$.}
We make the standard assumptions for constitutive functions  $g(\a) =(1-\a)^2$ and $h(\a) =\a^2$ and thus our model corresponds to the  phase field fracture model proposed  in  \cite{Bourdin2000-pc,Miehe2010-sj,Miehe2010-ja} referred to as the Ambrosio and Tortorelli (AT2) model (first introduced in~\cite{at}). We also choose the boundary conditions in the form: $u(0)=-\bar\epsilon/2$, $u(1)=\bar\epsilon/2$\comment[id=ALB]{`natural' bcs moved down} and we impose the displacement field  of the substrate   as $u^s(x)=\bar \epsilon/2 (2x-1)$.
\comment[id=ALB]{it would be practical to introduce a loading parameter $t$, scalar}
\added[id=ALB]{
Optimality conditions defining our problem are derived from the
intuitive idea that a state is stable (hence observable) if it is a local minimum of the energy  among admissible perturbations. In practice, for a given value of the control parameter $t$ we seek a state $y_t$ that satisfies}
\begin{equation}
    \label{eq:variational_global_ineq}
    \Psi(y_t) \leq \Psi(y_t + z),\quad \forall \text{ admissible } z
\end{equation}
\added[id=ALB]{where, in general, the admissibility of perturbations takes into account kinematic boundary conditions as well as other possible additional constraints.
For definiteness, provided a given initial condition $y_0$ at $t=0$, the quasi-static, incremental, evolutionary problem we address is to find a parametrised map $t \mapsto y_t \in X_t$, such that~\eqref{eq:variational_global_ineq} is satisfied. Here, $X_t$ is the ambient space for the state $y$ which possibly includes the control parameter in boundary conditions.
Solutions to the incremental problem are sought by solving first and second order necessary conditions for optimality.
}

The first variation of the energy functional is given \added[id=ALB]{by the following linear form}  
\begin{equation}
\delta \Psi(u,\alpha)(v,w)=\int_0^1
[g(\alpha)u'v'+\frac{1}{\lambda_2^2} (u-u_0) v+ \left( \frac{1}{2}u'^2 g'(\alpha)+h'(\alpha) \right)  w+\lambda_1^2\alpha'w' ]dx,\label{firstvar1}
\end{equation}
where $v$ and $w$ are scalar test functions. 
% Note here that, because the damage field $\alpha$ is free from irreversibility constraits, both positive and negative  test functions  are admissible, thus equilibrium configurations necessarily satisfy the equality to zero of the first order variation of the energy functional. 
By using standard arguments of the calculus of variations, we can localise the integral relation above to establish the strong (local) form of equilibrium conditions, and proceed with the evaluation of the equilibrium configuration via the Euler-Lagrange equations
\begin{eqnarray}
    % \label{modeld_el_1}
\begin{cases}
  2(1-\a)\a' u' +(1-\a)^2 u'' -  \frac{1}{\lambda_2^2}(u-u^s) &= 0, \\
  -\lambda_1^2\a'' - (1-\a)( u')^2 + 2\a   &= 0,
\end{cases}
\label{auto1}
\end{eqnarray}
\added[id=ALB]{and the associated the kinematic (essential) boundary conditions $u(0) = u^s(0), u(1) = u^s(1)$ as well as the natural conditions on the internal field $\alpha'(0)=\alpha'(1)=0$.
When the order parameter is reversible, the space of admissible perturbations is the homogeneous space associated to $V$, that is $V_0:= H^1_0((0,1))\times  H^1((0,1))$.
Remark that the choice of boundary conditions compatible with the substrate's deformation implies that $u(x)\equiv u^s(x)$ is (always) a solution to the elastic equilibrium equations with given homogeneous damage state $\alpha_h\in \mathbb{R}$. This makes it immediate to decouple the elasticity problem along homogeneous evolutions from the damage-driven stress softening.
}
Therefore, the homogeneous solution to~\eqref{auto1} is\deleted[id=ALB]{becomes for the damage  field}, 
\begin{equation}
u_h(x) =  \bar \epsilon/2 (2x-1),\qquad\alpha_h = \frac{\Bar{\epsilon}^2}{2 + \Bar\epsilon^2}\label{eq:homo1},
\end{equation}
and the effective \replaced[id=ALB]{total}{elastic} energy along the \replaced[id=ALB]{homogeneous}{trivial} branch reads  
$f_h(\bar\epsilon) :=\Psi(\alpha_h(\bar\epsilon), u_h(\bar \epsilon)) = \frac{\Bar{\epsilon}^2}{2 + \Bar\epsilon^2}$.
\added[id=ALB]{To study the stability of equilibria we distinguish two scenarios: i) that in which all small perturbations are admissible and belong to a (linear) vector space, and ii) the case where damage is subject to an irreversibility constraint, which results in a (nonlinear) system of inequalities.}

\added[id=ALB]{Damage criterion}

\subsection{Linear stability}
\comment[id=ALB]{How to understand linear/nonlinear}
An equilibrium configuration $y_t:=(u,\alpha)_t$, is a state such that the first variation $\delta \Psi(u,\alpha)\added[id=ALB]{(v, w)}$ vanishes for all admissible test fields \added[id=ALB]{in $V_0$} \comment[id=ALB]{for admissible $(v, w)$'s}. Incremental stability further requires\comment[id=ALB]{Notation suggestion: $w\mapsto \beta, y:=(u, \alpha), z:=(v, \beta)$}
\begin{equation}
\delta^2 \Psi(u,\alpha)(v,w)>0, \qquad  \forall (v,w)\in V_0,
\label{eqn:linear_second_order_stability}
\end{equation}
\replaced[id=ALB]{to be satisfied in for all directions $(v, w)\in V_0$}{compatible  fields $v$ and $w$,}. 
To assess the linear stability of the homogeneous solution, we examine the \added[id=ALB]{positivity of the} second variation \added[id=ALB]{which is given by the following bilinear form} \deleted[id=ALB]{as expressed by the following equation}:
\begin{equation}
\delta^2 \Psi(u,\alpha)(v,w)=\int_0^1 [(1-\alpha)^2v'^2 ,
%+ \lambda_2^{-2}v^2
- 4(1-\alpha)u' v'w+(2+ u'^2)w^2+\lambda_1^2w'^2 +\frac{1}{\lambda_2^2} v^2 ]dx,
\label{hessian22}
\end{equation}
well defined for perturbations $(v, w)\in V_0$.
\replaced[id=ALB]{We}{We then try to} extract information on the onset of instability expanding in Fourier series the fields $v$ and $w$ appearing in \eqref{hessian22} such that $v(x)=\sum_{n=1}^{\infty} a_{n} \sin \left(n \pi x+\phi_{n}\right), \quad w(x)=\sum_{n=1}^{\infty} b_{n} \cos \left(n \pi x+\psi_{n}\right)$. Then, we observe that, thanks to boundary conditions, $\psi_{n}=\phi_{n}=0$ for all natural $n$. The stability condition  takes the form:
\begin{align}\left[ a_n \quad b_n \right] \mathcal{H} \left[ \begin{array}{c} a_n \\ b_n \end{array} \right]=\left[ a_n \quad b_n \right]\left(
\begin{array}{cc}
g(\alpha)(n\pi)^2+\lambda_2^{-2}  & \frac{\partial g}{\partial \alpha}\bar\epsilon(n\pi)  \\
\frac{\partial g}{\partial \alpha}\bar\epsilon(n\pi)  &   \frac{\partial g^2}{\partial \alpha^2}\bar\epsilon^2+ \frac{\partial h^2}{\partial \alpha^2}+\lambda_1^2(n\pi)^2  \\
\end{array}
\right)\left[ \begin{array}{c} a_n \\ b_n \end{array} \right]>0.\label{hessian1}\end{align}
By substituting the homogeneous solution $\alpha_h$ into \eqref{hessian1}, we compute  the $\det \mathcal{H}$, as depicted  in Fig. \ref{fig:hessian1}. The calculations are performed for the parameter values \replaced[id=ALB]{$\lambda_1=0.16$}{$\lambda_1=0.158114$} and $\lambda_2=0.34$.  In Fig. \ref{fig:hessian1}(a), we observe the formation of closed loops, indicative of an elastic background's influence. Notably, a re-entry behavior of the affine configuration is discernible, marked by the emergence of two critical strains denoted as $\bar\epsilon^*$ and $\bar\epsilon^{**}$, representing the lower and upper stability limits for the homogeneous state. These critical points are highlighted by red and green dots in Fig. \ref{fig:hessian1}(a). The critical wavenumber $n_c$ for the lower limit $n_c(\bar{\epsilon}^*)$    differs from the critical wavenumber for  upper limit $n_c(\bar{\epsilon}^{**})$. Finally, we remark that closed-form analytical solutions can be provided for the critical wavenumber and critical strains and the parametric dependence of the corresponding bifurcation thresholds can be obtained, as detailed in \cite{Salman2021-mn}.


\begin{figure}
     \centering
     \includegraphics[scale=0.25]{./final_images/fig1.pdf}
\caption{
\todo[inline]{Add markers for $\bar \ep^{*}, \bar \ep^{**}$}
Computation of the determinant $\det \mathcal{H}$ for the homogeneous solution $\alpha_h$, utilizing parameter values \replaced[id=ALB]{$\lambda_1=0.16$}{$\lambda_1=0.158114$} and $\lambda_2 = 0.34$. The computation is performed for two scenarios: (a) rigid foundation and (b) deformable foundation with the additional parameter $r_s=0.5$. The closed loops observed indicate a re-entry behavior for large deformations. Black dots mark bifurcation from the homogeneous solution while red and green dots highlight the emergence of critical strains $\bar{\epsilon}^*$ and $\bar{\epsilon}^{**}$, representing the lower and upper stability limits, respectively. The critical wavenumber $n_c$ for the lower limit $n_c(\bar{\epsilon}^*)$ differs from that for the upper limit $n_c(\bar{\epsilon}^{**})$.}
     \label{fig:hessian1}
 \end{figure}


\subsection{{\added[id=ALB]{Nonlinear stability (with irreversibility constraints)}}}

\added[id=ALB]{
Irreversibility can be introduced in the variational formulation of the evolution problem as an inequality constraint. Intuitively, irreversibility plays two distinct roles along an evolution: (i) it is a local (pointwise) constraint, preventing the damage field at a given location to decrease between two subsequent load steps along smooth evolutions as well as through (stability) transitions, and (ii) it globally restricts the space of admissible variations in such a way that \emph{negative} perturbations of the current damage state are no longer allowed, effectively changing the structure of the set of admissible perturbations from a vector space to a convex cone. 
}
\added[id=ALB]{
To formulate irreversibility in the current model we consider non-decreasing damage evolutions that are sufficiently smooth with respect to the loading parameter, seeking maps $t\mapsto y_t = (u_t, \alpha_t)$ such that $\dot \alpha_t \geq 0$ and that satisfy a minimality condition~\eqref{eq:variational_global_ineq} 
Because the current state can only be compared to those of with damage equal or higher, the space of admissible perturbations becomes the set $K^+_0:=H^1_0((0, 1))\times \{w \in H^1((0, 1)): w(x) \geq 0 \text{ a.e. }x\in (0, 1)\}$.
By construction, the irreversible space of perturbations is contained in the vector space of unconstrained perturbations, namely $K^+_0\subseteq V_0$. Indeed, if $(0, \beta)\in K^+_0$, then $-(0, \beta)\notin K^+_0$. The latter property has a direct impact on the variational characterisation of local minima and bears consequences on both for the first order (equilibrium) conditions and the second order (stability) problem.}

\added[id=ALB]{Equilibrium states $y_t=(u_t, \alpha_t)$ of the irreversible system, namely such that $\dot \alpha_t(x) \geq 0 \text{ a.e. } x\in (0, 1)$, are hence governed by the following (first order) necessary optimality conditions taking the form of a variational inequality}
\begin{equation}
    \label{eq:eq_variational_inequality}
    \delta \Psi(y_t)(v-u_t, w-\alpha_t) \geq 0, \quad \forall
    % \delta \Psi(y)(v-u, w-\alpha) \geq 0, \quad \forall
    %  w -\alpha \in \{ w\in H^1(\Omega), w\geq 0\}
    (v-u_t, w - \alpha_t) \in K^+_{0}        
\end{equation}
% 
% where $K^+_{t}:=u^s(t) + H^1_0((0, 1))\times\{w \in H^1((0, 1)): w(x) \geq \alpha_t(x) \text{ a.e. }x\in (0, 1)\}$ depends on the loading parameter $t$ both explicitly, in the definition of kinematic boundary conditions, and implicitly, through the pointwise damage irreversibility.
% Notice that the main unknown, $\alpha$ is present both in the left-hand side of the inequality~\eqref{eq:variational_inequality} and in the definition of the admissible space of perturbations.
% 
from which follows 
\begin{equation}
    \label{eq:variational_equilibrium} 
    \delta \Psi(y_t)(v-u_t, 0) = 0, \qquad \delta \Psi(y_t)(0, w-\alpha_t) \geq 0, \qquad \forall (v-u_t, w-\alpha_t) \in K^+_{0}
\end{equation}
which are, respectively, the weak form of the mechanical equilibrium conditions and the threshold law for the evolution of the damage field.

\added[id=ALB]{The variational inequality satisfied by the damage field in~\eqref{eq:eq_variational_inequality} takes a particularly expressive form as a complementarity problem, 
which helps highlighting the mechanical nature of the damage criterion as a threshold law.
Namely, we seek $\alpha_t$ such that 
}
\begin{equation}
    \label{eq:complementarity}
    \dot \alpha_t \geq 0 \qquad 
     -\phi_t(\alpha_t) \geq 0 \qquad
     \phi_t(\alpha_t)\dot \alpha_t = 0
\end{equation}
where $\phi_t$ is the scalar function associated to the density of  released elastic energy, defined by $\delta \Psi(y_t)(0, \beta - \alpha_t) = \langle -\phi_t(\alpha_t), \beta - \alpha_t\rangle$.
%  Here, $\phi_t$ is the energy variation with respect to perturbations of the order parameter (from a given damage level $\alpha_t$) and is interpreted as an energy release rate. 
Consequently, $\phi_t(0)$ is the total energy released by the system in its sound state, and all equilibrium solutions $y_t$ such that $-\phi_t(\alpha_t) > 0$ are in the interior of the damage yield surface.
The equality to zero, conversely, defines the damage criterion as a yield surface.
The three conditions above encode the non-negativity of the damage rate, the boundedness of the elastic domain, and the complementarity between the attainment of the damage criterion and the evolution of the internal order parameter. 

------------

\added[id=ALB]{
In the current one-dimensional setup with homogeneous initial conditions $y_0=(0, 0)$ and compatible kinematic boundary conditions, the existence of a homogeneous solution implies that the damage criterion is attained homogeneously throughout the bar, which greatly simplifies the analysis of the energetic properties of the system. 
Using the elastic solution $u_t = 2t(x-1/2)$ and denoting by $\delta \alpha:=w-\alpha_t$ an admissible damage perturbation, the inequality in~\eqref{eq:variational_equilibrium} yields (by localisation) the following algebraic inequality
\begin{equation}
    \label{eq:variational_equilibrium_homogeneous}
    \phi_t(\alpha_t):= \frac{1}{2}t^2 g'(\alpha_t)+h'(\alpha_t) \geq 0
\end{equation}
which identifies the damage yield surface. 
The attainment of the equality in the last condition corresponds to the attainment of the critical energy release rate and which, in turn, implies the start of the evolution of the damage field.
We have $\delta \Psi(y_t)(0, \delta\alpha) =0, \quad\forall \delta \alpha \geq 0$ only if
$$
\frac{1}{2}t^2 g'(\alpha_t)+h'(\alpha_t) = 0, \qquad \text{a.e. } x\in (0, 1).
$$
The latter is an algebraic equation for $\alpha_t$ which satisfied by the evolution of an homogeneous damage corresponding to a given loading level $t$.
}


\added[id=ALB]{
Assume now that a state $y_t$ is known as a function of $t$ such that it solves~\eqref{eq:eq_variational_inequality}, and is sufficienty smooth so that the (right) derivative with respect to $t$ is well defined. 
As $t$ varies, $y_t$ describes a (smooth) curve in the phase space identified by its tangent vector $\dot y_t$ (and the initial condition).
A fundamental question is to discern whether $y_t$ is an isolated equilibrium describing a unique evolution path, or conversely it is the intersection of multiple equilibrium curves.
% two curves can intersect transversally or tangentially, there is an an- gular or tangent bifurcation following from the fact that the tangent directions to these curves are different or not at a bifurcation point.multiplicity
% The problem of finding all equilibrium curves going through a given equilibrium point is ... due to the nonlinearities of the problem. 
% Restricting our attention to evolutions which are smooth with respect to the loading parameter $t$, 
To this end, we can derive~\eqref{eq:eq_variational_inequality} with respect to $t$ to obtain a boundary value problem relating the rate of evolution $\dot y_t$ with the current state $y_t$, supposing the latter known, namely
% 
\begin{equation}
    \label{eq:variational_bifurcation}
    \text{find }\dot \zeta: \qquad \delta^2 \Psi(y_t)(\dot\zeta,  \zeta -\dot y_t) + \delta \dot \Psi(y_t)(\zeta-\dot y_t) \geq 0, \quad \forall \zeta \in K^+_{\dot y_t}
\end{equation}\comment[id=ALB]{explain the notation $K^+_{\dot y_t}$}
where $\delta \dot \Psi$ is the time-derivative of the linear form corresponding to the first order energy variation.
By construction, the homogeneous rate $\dot y_h$ is a solution, the question is whether another solution exists. The uniqueness is ensured by the positive definiteness of the quadratic form on the homogeneous linear space generated by $K^+_0$, namely $H^1_0((0,1)) \times H^1(\text{damaging})$. 
Thus, the non-bifurcation condition for the homogeneous evolution \comment[id=ALB]{because damaging=(0,1), explain}
}
% . The bifurcation problem is then governed by the following variational inequality
% - The bifurcation problem which is established by derivation of the first order equilibrium problem [.]
\begin{equation}
    \label{eq:variational_bifurcation}
    \delta^2 \Psi(y_t)(\dot\zeta, \dot\zeta) > 0, \quad \forall \dot\zeta \in H^1((0,1)) \times H^1(\text{damaging})    
\end{equation}
formally coincides with the classical linear stability problem of the reversible case~\eqref{eqn:linear_second_order_stability}, yet has a different mechanical interpretation in terms of evolution rates. 
The first load at which the latter inequality fails, namely $t_b:=\inf_t \{\delta^2 \Psi(y_t)(\zeta, \zeta) =0, \forall \zeta \in V_0 \}$ corresponds to the first critical load yielding the existence of (multiple) equilibrium curves through $y_{t_b}$ and thus the possibility of bifurcating away from homogeneous branch. The possibility of bifurcation is, however, not an energetic necessity.

Let's now consider the question of the stability of the homogeneous solution in the irreversible case. According to our energetic viewpoint, the stability of a state $y_t$ is governed by the positivity of $\delta^2 \Psi(y_t)$ on the constrained space of admissible equilibrium perturbations $K^+_0 \cap damaging$.
First, we observe that if the current equilibrium branch is unique then the current state is necessarily stable, yet the converse is not true.
By construction, the set inclusion $K^+_0 \cap damaging \subseteq V_0$ implies that, denoting $t_s$ the load at which the homogeneous soltution looses stability by analogy to the bifurcation load, namely $t_s: = \inf_t \{\delta^2 \Psi(y_t)(\zeta, \zeta) =0, \forall \zeta \in K^+_0 \cap damaging \}$, we have $t_b \leq t_s$.
This indicates a qualitative conceptual distinction between the bifurcation and the stability problems in the constrained case. As a consequence, a system can persist along a critical non-unique equilibrium branch, yet be stable. 
To assess the stability of a given state $y_t$ a necessary condition is given by the non-negativity of the hessian form on the constrained space of admissible perturbations, namely
% As a straightforward consequence,  
% Indeed, the critical load for loss of stability is conceptually different from the critical load at which the uniqueness of the (rate) response is lost, they are determined by two different problems which can be related by a (singular) asymptotic process.
\begin{equation}
     \label{eq:variational_stability}
     \delta^2 \Psi(y_t)(\zeta - y_t,  \zeta - y_t)  \geq 0, \quad \forall \zeta-y_t \in K^+_{0}
 \end{equation}
The solution of the variational ineqality above allows to characterise the stability of $y_t$, either yielding a positive eigenvalue $\lambda$ as a sufficient condition for the stability of current state, or a pair $(\kappa, z^*)\in \mathbb{R}^-\times K^+_0$ where $\kappa$ is the local (negative) curvature of maximum energy decrease and $z^*$ is interpreted as the \emph{instability mode} leading the system into an optimal direction of energy descent. 
From the numerical standpoint, the bifurcation eigen-problem in the vector space~\eqref{eq:variational_bifurcation} may be interpreted as a linearised or approximated version of the stability problem, where the irreversibility constraint is implemented as a pointwise inequality with respect to the previous time-step, cf.~\cite{Baldelli2021-gc}, without changing the structure of the perturbations space.
